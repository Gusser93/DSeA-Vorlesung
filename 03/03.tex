\chapter{Vorlesung}


\section{Landau-Notation}

$g, f : \mathbb{N} \rightarrow \mathbb{N}$

\subsection{$O(n)$}
\[g(n) \in O(f(n)) \Leftrightarrow c > 0 \land n_0 \in \mathbb{N}, \text{so dass für alle}~n \geq n_0~\text{gilt:} g(n) \leq c \cdot f(n) \Leftrightarrow \lim\limits_{n \rightarrow \infty}{sup \frac{g(n)}{f(n)}} < \infty\]
\paragraph{Beispiel}
\[\lim\limits_{n \rightarrow \infty}{\frac{n \log_2(n)}{n^2}} = \lim\limits_{n \rightarrow \infty}{\frac{log_2(n)}{n}} = \lim\limits_{n \rightarrow \infty} {\frac{\frac{\ln(n)}{\ln(2)}}{n}} \stackrel{\text{L' Hopital}}{=} \lim\limits_{n \rightarrow \infty}{\frac{1}{\ln(2)} \cdot \frac{1}{n}} = \frac{1}{\ln(2)} \lim\limits_{n \rightarrow \infty}{\frac{1}{n} = 0}
\]


\subsection{$\Omega(n)$}
\[g(n) \in \Omega(f(n)) \Leftrightarrow c > 0 \land n_0 \in \mathbb{N}, \text{so dass für alle}~n \geq n_0~\text{gilt:} g(n) \geq c \cdot f(n) \Leftrightarrow \lim\limits_{n \rightarrow \infty}{inf \frac{g(n)}{f(n)}} > 0\]
\paragraph{Beispiel} $g(n) = n^p~~~~f(n)=n^q~~~~p \geq q$
\paragraph{Behauptung} $g(n) \in \Omega(f(n))$

\[\lim\limits_{n \rightarrow \infty}{\frac{n^p}{n^q}} = \infty > 0\]


\subsection{$\Theta(n)$}
\[g(n) \in \Theta(f(n)) \Leftrightarrow g(n) \in O(f(n)) \land g(n) \in \Omega(f(n))\]
\paragraph{Beispiel} $g(n) = n^p + n^{p-1} + c \cdot n^2~~~~~f(n) = n^p$
\paragraph{Behauptung} $g(n) \in \Theta(f(n))$ \\
.... Rechnung

\subsection{$o(n)$}

\[g(n) \in o(f(n)) \Leftrightarrow \lim\limits_{n \rightarrow \infty}{\frac{g(n)}{f(n)}} = 0\]
\paragraph{Beispiel} $g(n) = n \cdot \log_2(n)~~~~f(n) = n^2$\\
\[\lim\limits_{n \rightarrow \infty}{\frac{g(n)}{f(n)}} = 0~~~\text{siehe oben}\]
\paragraph{Erklärung} ''g ist asymptotisch gesehen vernachlässigbar gegenüber f.''

\pagebreak


\begin{mdframed}
\subsection{Notation}
Häufig wird:
\[~~~~~~~~O(n) = O(n^2) = O(n^2 \cdot \log_2(n))\]
geschrieben, anstelle von:
\[~~~~~~~~O(n) \subset O(n^2) \subset O(n^2 \cdot \log_2(n))\]
Missbrauch der Notation !!!\\

\end{mdframed}


\section{Mergesort (Divide and Conquer)}

\subsection{Pseudo-Code}
\lstinputlisting[language=C, style=pseudo]{03/Code/mergesort.c}

\pagebreak

\subsection{Laufzeitanalyse}
$T(n) =$ Zahl der von Mergesort durchgeführten Elementarvergleiche $\approx$ Laufzeit 
\[T(n) = 2T(\frac{n}{2}) + n -1 \approx 2T(\frac{n}{2}) + n ~~~\text{mit}~T(1) = 0\]

\begin{mdframed}
\paragraph{Korrekter wäre} $T(n) = T(\lfloor \frac{n}{2}  \rfloor) + T(\lceil \frac{n}{2}  \rceil) + n -1~~~$ \hfill Für ungerade Zahlen \\

\end{mdframed}

\begin{flalign*}
 & T(n) = 2 \cdot 2 T \left(\frac{n}{2} \right) + n \overset{\text{(1)}}{=} 2 \left(\footnote{Unsicher ob diese Klammer, die linke, an der richtigen Stelle ist.} \left( 2 T \left(\frac{n}{4} \right)+ \frac{n}{2} \right) + n \right) = 4 T\left(\frac{n}{4} \right) + 2n& \\
 &~~~~~~~\stackrel{\text{(2)}}{=} 4 \cdot \left(2T \left(\frac{n}{8}\right) + \frac{n}{4} \right) + 2n = 8T\left(\frac{n}{8} \right) + 3n = ... = 2^i \cdot T \left(\frac{n}{2^i} \right)  + in& \\
 & & \\
 & T\left(\frac{n}{2} \right) = 2 T\left(\frac{n}{4}\right) + \frac{n}{2}~~~~~~(1)& \\
 & T\left(\frac{n}{4} \right) = 2 T\left(\frac{n}{8}\right) + \frac{n}{4}~~~~~~(2)& \\
 &....& \\
 & T(1) = 0& 
\end{flalign*}


\paragraph{Rekursionsende} $\frac{n}{2^i} = 1 \Leftrightarrow 2^i = n \Leftrightarrow i = \log_2(n)$\\
\begin{flalign*}
&T(n) = 2^{\log_2(n)} T\left(\frac{n}{2^{\log_2(n)}} \right) + n \log_2(n) = n T(1) + \log_2(n) = \log_2(n)&
\end{flalign*}


\paragraph{Abstraktion} \text{} \\
$T(n) =$ Laufzeit eines Divide \& Conquer Algorithmus der ein Problem dadurch löst, das es in $a$ Teilprobleme der Größe $\frac{n}{b}$ zerlegt wird, die rekursiv gelöst werden und anschließend kombiniert werden.\\

$T(n) = a\cdot T\left(\frac{n}{b} \right) + n^{\alpha} ~~~~~\alpha > 0~~~~~~\text{mit}~T(1) = 0$\\ 


\begin{flalign*}
&T(n) = a T\left(\frac{n}{b}\right) + n^{\alpha}  \stackrel{\text{(1)}}{=} a^2 T\left(\frac{n}{b^2}\right) + a\left(\frac{n}{b}\right)^{\alpha} + n^{\alpha}&\\
&(1)~~~~~T\left(\frac{n}{b}\right) = a T\left(\frac{n}{b^2}\right) + \left(\frac{n}{b}\right)^{\alpha}  \stackrel{\text{(2)}}{=} a^3 T\left(\frac{n}{b^3}\right) + a^2\left(\frac{n}{b^2}\right)^{\alpha}  + a^1\left(\frac{n}{b^1}\right)^{\alpha} +  a^0\left(\frac{n}{b^0}\right)^{\alpha}&\\
&(2)~~~~~T\left(\frac{n}{b^2}\right) = a T\left(\frac{n}{b^3}\right) + \left(\frac{n}{b^2}\right)^{\alpha}  = a^i T\left(\frac{n}{b^i}\right) + \sum_{j=0}^{i-1} a^j \left(\frac{n}{b^j}\right)^{\alpha} =  a^i T\left(\frac{n}{b^i}\right) + n^{\alpha} \sum_{j=0}^{i-1} \left(\frac{a}{b^{\alpha}}\right)^j&\\
&~~\text{mit}~i= \log_b(n) \land x = \frac{a}{b^{\alpha}}&\\
&...&\\
&T(1) = 0&
\end{flalign*}

\pagebreak
